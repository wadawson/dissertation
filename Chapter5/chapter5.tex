\newchapter{Perspective}{Perspective: Summary \& Discussion}{Perspective: Summary \& Discussion}
\label{chapter:5}

Chapter abstract text

\section{Dissertation Summary}

Over the past century our understanding of the universe has under dramatic revisions which have culminated in once inconceivably accurate (percent level) measurements of the composition of the universe.
While the general scientific community agrees upon the composition of the universe, the properties of the bulk of this composition (dark matter and dark energy) are a mystery.
This dissertation has presented our recent efforts to better understand the properties of dark matter (DM).

In Chapter \ref{chapter1} we provided a brief review of the history of DM (\S\ref{section:DMhistory}), showing that while Fritz Zwicky provided the first evidence of DM in 1933 \citep{Zwicky:1933ub} it wasn't until the work of Vera Rubin and collaborators \citep{Rubin:1970gu} in the 1970's that DM garnered much attention.
Three general candidates for DM originally dominated the debate.
One possibility was that it was simply massive compact objects made up of standard model particles, another was that it was some new particle, and another was the general relativity needed to be modified.
The MACHO experiment \citep{Alcock:2000bw} eventually ruled out the possibility that massive compact objects made up the bulk of DM, and merging galaxy clusters ruled out modified gravity \citep{Clowe:2006hr} and providing strong evidence for DM being a new particle.
We reviewed the generally accepted cold dark matter (CDM) properties (\S\ref{section:CDMproperties}) but provided motivation for considering that DM might actually interact with itself other than through gravity (SIDM model; \S\ref{section:SIDMmotivation}).
We reviewed the possible probes of SIDM (\S\ref{section:SIDMprobes}), highlighting  merging galaxy clusters {\S\ref{section:MergingClustersSIDMprobe}) and the four methods of constraining SIDM with observations of merging cluster.

In Chapter \ref{chapter:2} we introduced the the merging galaxy cluster DLSCL J0916.2+2951 (also known as the Musket Ball Cluster) and presented our multi-wavelength studies of the system.
Our photometric and spectroscopic observations show that the system consists of two subcluster separated by a projected distance of XXX and a redshift separation of XXX corresponding to a line-of-sight velocity difference of XXX.
Thus the two subclusters are close enough to be physically associated with one another.
Our weak lensing analysis of the two subclusters show that they have comparable mass XXX YYY, suggesting that they system is a major merger.
Our Sunyaev-Zel'dovich effect and X-ray observations show that the cluster gas is located between the two subclusters proving that the Musket Ball Cluster is a post merger system where the collisional gas has become dissociated from the effectively collisionless galaxies and DM.
Thus the Musket Ball Cluster is an excellent candidate to constrain the DM self-interaction cross-section ($\sigma_{\rm SIDM}$).

In Chapter \ref{chapter:3} we discussed the importance of understanding the dynamic history of mergers when attempting to use them to constrain the properties of DM.
We developed a new Monte Carlo based method to discern the properties of dissociative mergers and propagate the uncertainty of the measured cluster parameters in an accurate and Bayesian manner.
We verified it against an existing hydrodynamic N-body simulation, and applied it to two known dissociative mergers: 1ES 0657-558 (Bullet Cluster) and the Musket Ball Cluster.
We find that the dynamic properties of the Musket Ball represents a significantly different volume of merger phase space than the Bullet Cluster.
The Musket Ball Cluster, being $3.4^{+3.8}_{-1.4}$ times further progressed than the Bullet Cluster, could potentially provide tighter constraints on $\sigma_{\rm DM}$ since the offset between galaxies and dark matter should initially increase with time post-merger for  $\sigma_{\rm DM}>0$.

In Chapter \ref{chapter:4} we compare the locations of the galaxies, gas, and DM in the Musket Ball Cluster to provide insight into the properties of DM.
We are able to constrain the central gas distribution's projected centroid to within 9'' (57\,kpc at $z$=0.53), see \S\ref{section:GasLocation}.
Using both the extensive spectroscopic and photometric redshifts we are able to constrain the galaxy centroid of the northern subcluster to within 5.3" (33\,kpc at $z$=0.53) and the galaxy centroid of the southern subcluster to within 3.3" (21\,kpc at $z$=0.53), see \S\ref{section:GalaxyLocation}.
And using our tomographic WL method applied to the HST measured shapes we are able to constrain the projected WL centroid of the northern subcluster to within 13'' (82\,kpc at $z$=0.53) and the WL centroid of the southern subcluster to within 11'' (69\,kpc at $z$=0.53), see \S\ref{section:WLLocation}.
Our measurement of a significant offset of the gas between the DM in each subcluster enabled us to achieve the constraint $\sigma_{\rm DM} m_{\rm DM}^{-1} \lesssim 7$\,cm$^2$\,g$^{-1}$.
Given the dependence on the surface mass density of this method it is not surprising that this constraint is less than that achieved with more massive mergers \citep{Markevitch:2004dl, Bradac:2008gw, Merten:2011gu}.
Finally in this chapter we investigate the galaxy-WL offset since if DM self-interacts then the effectively collisionless galaxies might be expected to lead the DM post-merger.
While we find that the galaxies appear to be leading the WL centroid in the southern subcluster by $\sim$20.5'' (129\,kpc at $z$=0.53), see \S\ref{section:GalaxyWLOffset}, this only provides $\sim$85\% confidence that $\sigma_{\rm DM}>0$.
Furthermore when we account for the observation that the galaxy centroid appears to trail the WL centroid in the northern subcluster by $\sim$7.4'' (47\,kpc at $z$=0.53), the confidence that $\sigma_{\rm DM}>0$ falls to $\sim$55\%.
While the SIDM scenario is slightly preferred over the CDM scenario it is not significantly so.
SIDM simulations of the Musket Ball Cluster are needed to turn these observations into quantitative constraints on $\sigma_{\rm DM}$.


\section{Discussion on the Path Forward}






need to discuss the feasibility of this approach in light of \citet{Kahlhoefer:2013wp} and the centroid uncertainties measured in this dissertation.

Discuss work of \citep{Kahlhoefer:2013wp}
- note that observed south offset is far larger than that predicted by their recent simulation, what about average of two offsets (20.5, -7.44) = 6.5' = 41 kpc
- if there work is correct then unlikely to make this measurement given the calculated galaxy and WL centroid accuracies XXX\,kpc and YYY\,kpc, respectively.





Talk about Subaru WL as more evidence that south only is an optimistic


What are the major challenges to making merging clusters the best probes of SIDM
- intrinsic scatter in the galaxy-WL offset
- intrinsic scatter in the M/L ratio
-- look for assymetries in the M/L ratio
- small magnitude of the theoretically predicted offset





% text copied from my fellowship applications:
\subsection{Bridging the Gap Between Observations and Simulations}

\textit{Should look at some proposal text}

The SIDM simulations of observed mergers are one of the most important aspects of the MC2 plan.  Not only are they necessary to place the tightest constraints on $\sigma$DM  (e.g. Randall et al. 2008), but by applying the same measurement techniques to the simulations as the observed mergers they will enable us to marginalize over systematic errors.  

Beyond the computational challenges of simulating SIDM (which our group has recently mastered, Rocha et al. 2012 \& Peter et al. 2012), it is highly non-trivial to simulate an actual observed merger. This is due to the fundamentally limited information observations provide for a given merger (Dawson 2012b). Thus a single observed merger can conceivably be represented by a wide range of simulated merger scenarios. To address this issue I will implement an importance sampling method to identify likely realizations of the observed merger in cosmological N-body simulations (Figure 3), extract a representative sample of these mergers, and in collaboration with MC2 members resimulate these with SIDM physics (Dawson et al. in prep).  Because each simulated realization of the observed merger will have an associated likelihood, I will be able to use the resulting $\sigma$DM  constraints of each realization to create a posterior probability density function.  This work will result in the first quantitative estimate of the $\sigma$DM  constraint uncertainty. 

\subsection{Theoretical model of the behavior of SIDM during mergers}

Develop a theoretical model of the behavior of SIDM during the merger process and implement this into my existing analytic merger code. 

While Project 1 will enable us to extract the tightest possible $\sigma$DM  constraints from a given merger, it is computationally intensive and marginalizes over interesting physical phenomena.  I am working with Manoj Kaplinghat to develop a theoretical model of the merger process involving SIDM.  I will then incorporate this model into my existing method for determining the dynamic properties of observed mergers (Dawson 2012b), enabling quantitative measurement or constraint of $\sigma$DM  for a given merger and associated galaxy-DM offset.  This project is highly complementary to Project 1: firstly, comparing the predictions of this analytic method with the more detailed simulations will provided needed confirmation and secondly, since simulations are expensive this quick analytic method can be used to determine the mergers with the greatest constraining power and prioritize their position in the simulation queue.

\subsection{Simulating mergers with SIDM}

Use results of the SIDM simulations of multiple observed dissociative mergers to place the best measurement or tightest constraint possible on $\sigma$DM.  

Because I will have taken care to properly quantify the uncertainty of dark matter constraints from each of the dissociative mergers (see Project 1), I will be able to properly combine the weighted expectations of each merger’s $\sigma$DM  constraint to effectively beat down the Poisson noise of the centroid offset measurement.   It is only through a consistent and systematic approach (such as the proposed) that one can properly combine the constraints of individual mergers. 

\subsection{Find and study more dissociative mergers}

\subsubsection{Ongoing efforts}

Radio relic sample and existing mergers.

\subsubsection{Near-term}

More radio relics and SZ+DES mergers

\subsubsection{Next ten years}

LSST + all sky X-ray surveys (e.g. eROSITA)

%text copied from fellowship applications
Identify and follow-up more dissociative mergers to increase the sample size and further reduce the random noise of the centroid measurement in order to constrain more complex SIDM models (e.g. velocity dependent cross-section).

I will refine the optical-SZ identification method with plans of applying it to the overlapping DES and ACT/SPT surveys.  Based on the simulated and observed SZ cluster counts in the SPT survey (Vanderlinde et al. 2010 \& Song et al. 2012) and the fraction of all clusters that are dissociative mergers (Forero-Romero et al. 2010) I calculate that $\sim$50 detectable dissociative mergers will be observed in the 4000 deg2 SPT-DES survey.  Selection bias will result in mergers with large mass (i.e. better signal-to-noise) and large projected separations (i.e. later stage mergers where the expected DM-galaxy offset is maximized).  These will be some of the best dissociative mergers with which to constrain $\sigma$DM .

Together with Reinout van Weeren (Einstein Fellow at CfA and member of MC2), we are pioneering a method of dissociative merger identification using radio observations.  Radio selection is a potentially efficient way to select merging systems because the shock produced by a merger results in a ``radio relic'' (diffuse emission in an arc around part of the cluster, e.g. van Weeren et al. 2010) and wide-area radio surveys will provide many candidates. NVSS has already found a few dozen, and LOFAR is projected to find about 1000 (Nuza et al. 2012).

Overall this project will provide large samples from which to cull the very best mergers (complement of Project 2) and follow-up with detailed observations and simulations.  Additionally, the sheer numbers involved will make it feasible to perform the galaxy-DM offset measurement using ground based data alone (e.g. Magellan).

\section{Current Progress along this Path}

Section text


\textbf{acknowledgements:}
Acknowledgment text
%\end{acknowledgements}

%\bibliographystyle{apj}
%\bibliography{Chapter1/chapter1}{}



%% The References
%\bibliographystyle{thesis}
%\begin{singlespacing}
%  \bibliography{Chapter3/chapter3}
%\end{singlespacing}
