\newchapter{Perspective}{Perspective: Summary \& Discussion}{Perspective: Summary \& Discussion}
\label{chapter:5}

Chapter abstract text

\section{Dissertation Summary}

Intro text

\section{Discussion on the Path Forward}

% text copied from my fellowship applications:
\subsection{Bridging the Gap Between Observations and Simulations}

\textit{Should look at some proposal text}

The SIDM simulations of observed mergers are one of the most important aspects of the MC2 plan.  Not only are they necessary to place the tightest constraints on $\sigma$DM  (e.g. Randall et al. 2008), but by applying the same measurement techniques to the simulations as the observed mergers they will enable us to marginalize over systematic errors.  

Beyond the computational challenges of simulating SIDM (which our group has recently mastered, Rocha et al. 2012 \& Peter et al. 2012), it is highly non-trivial to simulate an actual observed merger. This is due to the fundamentally limited information observations provide for a given merger (Dawson 2012b). Thus a single observed merger can conceivably be represented by a wide range of simulated merger scenarios. To address this issue I will implement an importance sampling method to identify likely realizations of the observed merger in cosmological N-body simulations (Figure 3), extract a representative sample of these mergers, and in collaboration with MC2 members resimulate these with SIDM physics (Dawson et al. in prep).  Because each simulated realization of the observed merger will have an associated likelihood, I will be able to use the resulting $\sigma$DM  constraints of each realization to create a posterior probability density function.  This work will result in the first quantitative estimate of the $\sigma$DM  constraint uncertainty. 

\subsection{Theoretical model of the behavior of SIDM during mergers}

Develop a theoretical model of the behavior of SIDM during the merger process and implement this into my existing analytic merger code. 

While Project 1 will enable us to extract the tightest possible $\sigma$DM  constraints from a given merger, it is computationally intensive and marginalizes over interesting physical phenomena.  I am working with Manoj Kaplinghat to develop a theoretical model of the merger process involving SIDM.  I will then incorporate this model into my existing method for determining the dynamic properties of observed mergers (Dawson 2012b), enabling quantitative measurement or constraint of $\sigma$DM  for a given merger and associated galaxy-DM offset.  This project is highly complementary to Project 1: firstly, comparing the predictions of this analytic method with the more detailed simulations will provided needed confirmation and secondly, since simulations are expensive this quick analytic method can be used to determine the mergers with the greatest constraining power and prioritize their position in the simulation queue.

\subsection{Simulating mergers with SIDM}

Use results of the SIDM simulations of multiple observed dissociative mergers to place the best measurement or tightest constraint possible on $\sigma$DM.  

Because I will have taken care to properly quantify the uncertainty of dark matter constraints from each of the dissociative mergers (see Project 1), I will be able to properly combine the weighted expectations of each merger’s $\sigma$DM  constraint to effectively beat down the Poisson noise of the centroid offset measurement.   It is only through a consistent and systematic approach (such as the proposed) that one can properly combine the constraints of individual mergers. 

\subsection{Find and study more dissociative mergers}

\subsubsection{Ongoing efforts}

Radio relic sample and existing mergers.

\subsubsection{Near-term}

More radio relics and SZ+DES mergers

\subsubsection{Next ten years}

LSST + all sky X-ray surveys (e.g. eROSITA)

%text copied from fellowship applications
Identify and follow-up more dissociative mergers to increase the sample size and further reduce the random noise of the centroid measurement in order to constrain more complex SIDM models (e.g. velocity dependent cross-section).

I will refine the optical-SZ identification method with plans of applying it to the overlapping DES and ACT/SPT surveys.  Based on the simulated and observed SZ cluster counts in the SPT survey (Vanderlinde et al. 2010 \& Song et al. 2012) and the fraction of all clusters that are dissociative mergers (Forero-Romero et al. 2010) I calculate that $\sim$50 detectable dissociative mergers will be observed in the 4000 deg2 SPT-DES survey.  Selection bias will result in mergers with large mass (i.e. better signal-to-noise) and large projected separations (i.e. later stage mergers where the expected DM-galaxy offset is maximized).  These will be some of the best dissociative mergers with which to constrain $\sigma$DM .

Together with Reinout van Weeren (Einstein Fellow at CfA and member of MC2), we are pioneering a method of dissociative merger identification using radio observations.  Radio selection is a potentially efficient way to select merging systems because the shock produced by a merger results in a ``radio relic'' (diffuse emission in an arc around part of the cluster, e.g. van Weeren et al. 2010) and wide-area radio surveys will provide many candidates. NVSS has already found a few dozen, and LOFAR is projected to find about 1000 (Nuza et al. 2012).

Overall this project will provide large samples from which to cull the very best mergers (complement of Project 2) and follow-up with detailed observations and simulations.  Additionally, the sheer numbers involved will make it feasible to perform the galaxy-DM offset measurement using ground based data alone (e.g. Magellan).

\section{Current Progress along this Path}

Section text


\textbf{acknowledgements:}
Acknowledgment text
%\end{acknowledgements}

%\bibliographystyle{apj}
%\bibliography{Chapter1/chapter1}{}



%% The References
%\bibliographystyle{thesis}
%\begin{singlespacing}
%  \bibliography{Chapter3/chapter3}
%\end{singlespacing}
