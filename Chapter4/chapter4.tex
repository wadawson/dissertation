\newchapter{Musket Ball: DM Implications}{Musket Ball Cluster: Dark Matter Implications}{Musket Ball Cluster: Dark Matter Implications}
\label{chapter:4}

\noindent Portions of this chapter were originally published in the article titled \emph{Discovery of a Dissociative Galaxy Cluster Merger with Large Physics Separation} which was published in the March 2012 issue of the Astrophysical Journal Letters (Volume 747, pp. L42). \\

Chapter abstract text

\section{Introduction}



Intro text \citep{Dawson:2012dl}

\section{Location Estimation}
For the original work of \citet{Dawson:2012dl} (see Chapter 2) I estimated the weak lensing subcluster positions and errors on their positions by using the location of the peak signal in a region and estimated the variance on the peak by measuring the peak location of each bootstrap iteration within the selected region.  
For this chapter, rather than using the location of the peak and variance of the peak, I have adopted an iterative centroid estimation scheme, similar to \citet{Randall:2008hs}. 
I begin by calculating the centroid of a large aperture that encompasses one subcluster, but excludes the other.
I then decrease the aperture, recenter on the previously calculated centroid, and estimate the centroid of the new aperture.
This process is repeated until the aperture is decreased to a radius of XXX kpc.
To estimate the uncertainty on the location I perform this process on each iteration of a random bootstrap sample map, resulting in a array of centroid values.
 The uncertainty on the location is then inferred from the variance of this array of centroid values.


\subsection{Galaxies}
\subsection{Gas}
\subsection{Weak Lensing}


\section{Gas--Weak Lensing Offset}

%copied from \citep{Dawson:2012dl}
Given the evident merger scenario we are able to use the first method of \citet{Markevitch:2004dl} and place a rough limit on the DM self-interaction cross-section, $\sigma_{\rm DM}$.
This method compares the scattering depth of the dark matter, $\tau_{\rm DM}=\sigma_{\rm DM}m^{-1}_{\rm DM} \Sigma_{\rm DM}$, with that of the ICM gas, $\tau_{\rm ICM}\approx 1$, where $m_{\rm DM}$ is the DM particle mass and $\Sigma_{\rm DM}$ is the surface mass density of the DM particles.
$\Sigma_{\rm DM}$ is approximately the WL measured surface mass density, $\Sigma$, since $\sim80\%$ of a typical cluster's mass is DM \citep{Diaferio:2008js}.
For ease of comparison with the results of \citet{Markevitch:2004dl} and \citet{Merten:2011gu} we examine the surface density averaged over the face of the subcluster within $r$=125\,kpc, which is $\Sigma\approx0.15$\,g\,cm$^{-2}$; thus we find $\sigma_{\rm DM} m_{\rm DM}^{-1} \lesssim 7$\,cm$^2$\,g$^{-1}$.  
Note that we cannot apply the velocity-dependent $\sigma_{\rm DM}$ constraint methods outlined by \citet{Markevitch:2004dl} since our analytic model assumes $\sigma_{\rm DM}$\,=\,0 \citep{Dawson:2012dl, Dawson:2012ub}.

\begin{figure}
%\plottwo{fig3a.eps}{fig3b.eps}
\plotone{Chapter2/fig3.png}
\caption{Comparison of the Subaru $i'$-band ground-based (left) and HST space-based (right) WL mass signal-to-noise maps (color) of DLSCL J0916.2+2951 with the X-ray distribution (bold black contours) and galaxy number density (white contours, same as Figure \ref{fig2}). The peak centers and corresponding one sigma errors are denoted by the gray cross-hairs.
In both analyses there is agreement between the location and relative magnitude of galaxies and WL yet the majority of the cluster gas is centered $\sim1.4\arcmin$ between the North and South subclusters in a local mass underdensity, providing evidence that the North and South subclusters have undergone the first pass-through of a major merger.
The scale of each map is equivalent and the image field-of-view is the same as Figures \ref{fig1} \& \ref{fig2}.
The map created from the joint Subaru/HST catalog looks nearly identical to the HST map, with only slight variations in the scale (see Table \ref{tbl1}).
\label{fig3}}
\end{figure}

\section{Galaxy--Weak Lensing Offset}

Section text

\section{Discussion}

Discussion text

\subsection{Subsection title}

\section{Conclusions}

Conclusion text

\textbf{acknowledgements:}
Acknowledgment text
%\end{acknowledgements}

%\bibliographystyle{apj}
%\bibliography{Chapter1/chapter1}{}



%% The References
%\bibliographystyle{thesis}
%\begin{singlespacing}
%  \bibliography{Chapter3/chapter3}
%\end{singlespacing}
