\newchapter{Introduction}{Introduction}{An Introduction to Self-interacting Dark Matter and Merging Galaxy Clusters}
\label{chapter:1}

Chapter abstract text

%--------------------------------------------------------------------
\section{Composition of the Universe}\label{section:CompositionOfTheUniverse}

Over the past one-hundred years there has been a revolution in our understanding of the universe.
From believing that the Milky Way was the extent of the Universe, the age of Universe was infinite, and that the Universe was composed entirely of atomic matter and photons (cite Hoyle 1950 book).
To now having evidence that the Universe at least extends as far as light can travel in the finite age of the universe ($\sim XXX \pm YYY$), and is dominated by mysterious dark energy and dark matter rather than atomic matter.
Our studies of the universe have improved to such a degree that we are now able to quantify the composition of the universe to percent level accuracy. 
We find that atomic matter (which is dominated by baryonic particles, i.e. particles of the atomic nuclei) only accounts for $\sim$5\% of the universe's total matter-energy budget.
The universe appears to largely be composed of dark matter (DM; $\sim$26\%) and dark energy \citep[$\sim$69\%; see][for more accurate values]{Collaboration:2013uv}.
Very little is known about each of these components.
In very basic terms, dark energy appears to be a particle or field that acts to make the universe expand, while DM appears to be a particle that currently has only been observed to interact via the gravitational force.
This dissertation will focus on our efforts to improve our understanding of DM, in particular to ascertain whether DM interacts with itself other than through the gravitational force.

%--------------------------------------------------------------------
\section{Dark Matter}

\subsection{Historical Review}

``Dunkle (kalte) Materie'' (or ``Dark (cold) Matter'') was first proposed by Fritz Zwicky in 1933 \citep{Zwicky:1933ub}.
Zwicky had measured the velocities of the galaxies in the Coma galaxy cluster, via their spectroscopic redshifts, and found that given the large velocities the galaxies should not be gravitationally bound if galaxies/stars make up the entirety of the cluster mass.
According to his calculations the total cold dark matter (CDM) mass must be about 400 times that of the mass of the visible galaxies. 
As \citet{vandenBergh:1999jf} notes in his review article, had Zwicky used the correct value for the Hubble constant (rather than H$_0$=558\,km\,s$^{-1}$\,Mpc$^{-1}$) he would have found that the CDM mass must be about 50 times that of the mass of the luminous matter, very close to the actual value of $\sim20$ (disregarding the mass of the X-ray emitting intracluster gas that was unbeknownst to astronomers at the time).
Despite Zwicky's pioneering and shocking findings, the idea of DM did not garner much attention until the work of \citet{Rubin:1970gu} on the rotation velocities of stars and gas in spiral galaxies (e.g. the Andromeda galaxy).
This work was very much in the same vein as Zwicky's work on the Coma galaxy cluster. 
Interestingly, Zwicky also introduced another, completely independent, method of measuring the mass of galaxy clusters and coined the term ``gravitational lens effect'' \citep{Zwicky:1937ec}.
Zwicky argued that if his mass estimates of the Coma cluster were correct, then the cluster should be massive enough to distort space-time to such a degree that as light from distant background galaxies travels through the gravitational potential well of the cluster the light will be deflected and the galaxy images will be distorted in a coherent fashion.
Thus by measuring the distortion of galaxies behind a galaxy cluster one could estimate the mass of the galaxy cluster \citep[see Chapter 2 of][for an introduction to weak gravitational lensing]{Courbin:2002wh}.
\citet{Tyson:1990bc} were the first to successfully measure the gravitational lens effect of a galaxy cluster.
They too found that a mass of the luminous matter alone was insufficient to produce the observed effect.

While the aforementioned work, and subsequent work, found that amount of luminous mass galaxies and clusters was not enough to gravitationally bind the stars and galaxies in those structure, there was still a debate about the solution to this problem. 
Two possible solutions dominated the debate: Zwicky was correct and there existed DM particles, or our understanding of the gravitational force needed to be modified \citep[see][for a review]{Sanders:2002cc}.
This debate remained one of the largest unsettled debates in physics and astronomy until \citet{Clowe:2004eq} published their studies of the Bullet Cluster (1E 0657-558).
The Bullet Cluster was the first discovered \textit{dissociative merger}, see Figure \ref{fig:MergerTimeSeries} for an example of a dissociative merger.
During the merging process the effectively collisionless galaxies  become dissociated from the collisional intracluster gas, which strongly interacts during the merger and becomes pancaked at the point of collision.
In total the cluster gas is about seven times as massive as the gas.
When \citet{Clowe:2004eq} measured where the total cluster mass was with the gravitational lens effect they found that the majority of the mass was located with the galaxies rather than with the gas.
They then inferred that this was only possible if there was a DM particle that was nearly collisionless like the galaxies; modified gravity could not easily explain such an observation.

\subsection{General Properties of Dark Matter}

For a nice review of our current understanding of DM I recommend \citet{Peter:2012tg}, however I will summarize a few relevant points here.
In addition to the abundance of DM (see \S \ref{section:CompositionOfTheUniverse}) there are a few things we know about the properties of DM.
Most notably DM is electromagnetically neutral.
DM does not interact with photons, either through absorption or emission.
Baryons cannot make up a large portion of DM.
This is know from observations of the cosmic microwave background, large-scale structure of the universe, and abundance of light elements created during big-bang nucleosynthesis.
Nor can DM consist solely of light (sub-keV-mass) particles (e.g. neutrinos). 
Light particles move too fast in the early universe to form the initial density concentrations necessary to seed the cosmic structures observed at later times.
DM is completely outside the realm our Standard Particle Physics Model.
Thus there is no reason to expect that it only interacts via the known forces.
It is entirely possible that DM interacts with itself via some new dark gauge bosons.
Such DM has been termed self-interacting dark matter (SIDM).

%--------------------------------------------------------------------
\section{Motivation for Studying Self-interacting Dark Matter}

\subsection{Early Motivation}

Despite the success of the CDM paradigm there are a number of recent observations which appear to be in conflict:

Earliest was the missing satellites problem \citep{Moore:1999ja} \citep{Klypin:1999ej}

	Motivated work of Spergel and Steinhardt into SIDM \citep{Spergel:2000cb}

	Recent work shows that doesn't SIDM doesn't significantly reduce the number of satellites

Has since become less of a problem (simulations and SDSS)

\subsection{Early SIDM Constraints}

Halo Shapes: Miralda-Escudé 2002

Cluster density core: Yoshidal et al. 2000

Dwarfs density cores: Dave et al. 2001

Subhalo evaporation: Gnedin \& Ostriker 2001

Merging Cluster: Randall et al 2008

Growth of SMBH: Hennawi \& Ostriker 2002

Moore et al. 2000 \citep{Moore:2000ee}

Problems with each of those (see presentation slides)

Recent revisions to these constraints \citep{Peter:2012vi}

\subsection{Recent Motivation}

Cusp-core problem

	rachael kuza de neray
	drew newman
	counter example: Tomasso's student, Louie Strigari (see KITP presentation)

Too-big to fail

\subsection{SIDM as a Solution}

How SIMD may resolve these problems

Narrow window of parameter space to probe \citep{Rocha:2012tr}

%--------------------------------------------------------------------
\section{Probes of SIDM}



Baryon degeneracy and why merging cluster may be the best probe

%--------------------------------------------------------------------
\section{Merging Galaxy Clusters as Probes of Self-Interacting Dark Matter}

Section text

\subsection{Merging Galaxy Clusters}\label{Section:MergingClusters}

Think about moving some of the Figure \ref{fig:4ConstraintMethods} caption to the body text to make for a more concise caption.

\begin{figure}
\centering
\includegraphics[width=6in]{Chapter1/MergerTimeSeries.png}
\caption{A basic time series leading to a dissociative galaxy cluster merger (assuming the Standard Model of Cosmology).
Two galaxy clusters (N \& S), each consisting of overlapping halos of DM and gas as well as sparsely populated galaxies, begin with an initial physical separation ($t_{\rm 1}$).
Due to the mass of each galaxy cluster they experience a gravitational attraction and accelerate towards one another until eventually they collide ($t_{\rm 1}$--$t_{\rm 4}$).
By convention time $t_{\rm 4}$ is defined as the ``collision''.
Because there is so much space between the galaxies, a strong interaction between any two galaxies is extremely unlikely and they can be treated as effectively collisionless particles.
And since the galaxies of each subcluster have built up momentum they will pass through and begin to separate (this time on the opposite side), $t_{\rm 5}$.
The gas however is more evenly distributed, thus an interaction between gas particles of each subcluster is more likely.
This interactions will convert some of the infall kinetic energy into thermal energy (i.e. the gas of each subcluster will experience ram pressure), the net effect being that the gas halo of each subcluster is slowed with respect to the galaxies and much of it becomes dissociated remains centered between the galaxies of the two subclusters ($t_{\rm 6}$ C).
Much like the galaxies the DM behaves in a nearly collisionless manner and appears largely coincident with the galaxies.
At $t_{\rm 6}$ the galaxy cluster merger is classed as \emph{dissociative merger}.
\label{fig:MergerTimeSeries}}
\end{figure}  

\subsection{Constraining Self-Interacting Dark Matter with Merging Galaxy Clusters}

Think about moving some of the Figure \ref{fig:4ConstraintMethods} caption to the body text to make for a more concise caption.

\begin{figure}
\centering
\includegraphics[width=4in]{Chapter1/4ConstraintMethods.png}
\caption{The four means of constraining $\sigma_{\rm SIDM}$ with merging clusters, originally outlined by \citet{Markevitch:2004dl} and \citet{Randall:2008hs}. 
\emph{Upper Left:} If the DM is significantly offset from the gas then the scattering depth of the dark matter ($\tau_{\rm DM}$) must be less than the scattering depth of the gas ($\tau_{\rm gas}$) and an upper limit can be placed on $\sigma_{\rm SIDM}$.
\emph{Upper Right:} If DM self-interacts during the merger, then the velocity of each subcluster will be slowed to some degree.
Thus if the observed velocity ($v_{\rm obs}$) is found to be consistent with the free-fall velocity ($v_{\rm free-fall}$) an upper limit can be placed on $\sigma_{\rm SIDM}$.
If however $v_{\rm obs}$ is significantly less than $v_{\rm free-fall}$, an upper limit could be potentially placed on $\sigma_{\rm SIDM}$.
\emph{Lower Left:} If DM self-interacts during the merger, then some fraction of the DM particles will scatter and become unbound from each subcluster.
Thus the mass-to-light ratio of each subcluster can be compared with the mass-to-light ratio of similar non-merging clusters, and depending on where the mass-to-light ratio of the merger is the same or less than the non-merging clusters' mass-to-light ratio, then respectively an upper limit or lower limit can be placed on $\sigma_{\rm SIDM}$.
\emph{Lower Right:} If the DM self-interacts during the merger, then the DM component of each subcluster will experience an additional drag force and will travel at a slower velocity than the respective subcluster galaxies.
Thus depending on whether a significant offset between the galaxies and DM is or is not observed, then respectively a lower limit or upper limit can be placed on $\sigma_{\rm SIDM}$.
\label{fig:4ConstraintMethods}}
\end{figure}

\section{Work Presented in this Dissertation}
Briefly summarize each chapter.