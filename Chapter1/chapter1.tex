\newchapter{Introduction}{Introduction}{An Introduction to Self-interacting Dark Matter and Merging Galaxy Clusters}
\label{chapter:1}

Chapter abstract text

%--------------------------------------------------------------------
\section{Composition of the Universe}\label{section:CompositionOfTheUniverse}

Over the past one-hundred years there has been a revolution in our understanding of the universe.
From believing that the Milky Way was the extent of the Universe, the age of Universe was infinite, and that the Universe was composed entirely of atomic matter and photons (cite Hoyle 1950 book).
To now having evidence that the Universe at least extends as far as light can travel in the finite age of the universe ($\sim XXX \pm YYY$), and is dominated by mysterious dark energy and dark matter rather than atomic matter.
Our studies of the universe have improved to such a degree that we are now able to quantify the composition of the universe to percent level accuracy. 
We find that atomic matter (which is dominated by baryonic particles, i.e. particles of the atomic nuclei) only accounts for $\sim$5\% of the universe's total matter-energy budget.
The universe appears to largely be composed of dark matter (DM; $\sim$26\%) and dark energy \citep[$\sim$69\%; see][for more accurate values]{Collaboration:2013uv}.
Very little is known about each of these components.
In very basic terms, dark energy appears to be a particle or field that acts to make the universe expand, while DM appears to be a particle that currently has only been observed to interact via the gravitational force (CDM).
This has come be known as the concordance cosmological model ($\Lambda$CDM).
This dissertation will focus on our efforts to further our understanding of DM, in particular to ascertain whether DM interacts with itself other than through the gravitational force or if it really is just CDM.

%--------------------------------------------------------------------
\section{Dark Matter}

\subsection{Historical Review}

``Dunkle (kalte) Materie'' (or ``Dark (cold) Matter'') was first proposed by Fritz Zwicky in 1933 \citep{Zwicky:1933ub}.
Zwicky had measured the velocities of the galaxies in the Coma galaxy cluster, via their spectroscopic redshifts, and found that given the large velocities the galaxies should not be gravitationally bound if galaxies/stars make up the entirety of the cluster mass.
According to his calculations the total cold dark matter (CDM) mass must be about 400 times that of the mass of the visible galaxies. 
As \citet{vandenBergh:1999jf} notes in his review article, had Zwicky used the correct value for the Hubble constant (rather than H$_0$=558\,km\,s$^{-1}$\,Mpc$^{-1}$) he would have found that the CDM mass must be about 50 times that of the mass of the luminous matter, very close to the actual value of $\sim20$ (disregarding the mass of the X-ray emitting intracluster gas that was unbeknownst to astronomers at the time).
Despite Zwicky's pioneering and shocking findings, the idea of DM did not garner much attention until the work of \citet{Rubin:1970gu} on the rotation velocities of stars and gas in spiral galaxies (e.g. the Andromeda galaxy).
This work was very much in the same vein as Zwicky's work on the Coma galaxy cluster. 
Interestingly, Zwicky also introduced another, completely independent, method of measuring the mass of galaxy clusters and coined the term ``gravitational lens effect'' \citep{Zwicky:1937ec}.
Zwicky argued that if his mass estimates of the Coma cluster were correct, then the cluster should be massive enough to distort space-time to such a degree that as light from distant background galaxies travels through the gravitational potential well of the cluster the light will be deflected and the galaxy images will be distorted in a coherent fashion.
Thus by measuring the distortion of galaxies behind a galaxy cluster one could estimate the mass of the galaxy cluster \citep[see Chapter 2 of][for an introduction to weak gravitational lensing]{Courbin:2002wh}.
\citet{Tyson:1990bc} were the first to successfully measure the gravitational lens effect of a galaxy cluster.
They too found that a mass of the luminous matter alone was insufficient to produce the observed effect.

While the aforementioned work, and subsequent work, found that amount of luminous mass galaxies and clusters was not enough to gravitationally bind the stars and galaxies in those structure, there was still a debate about the solution to this problem. 
Two possible solutions dominated the debate: Zwicky was correct and there existed DM particles, or our understanding of the gravitational force needed to be modified \citep[see][for a review]{Sanders:2002cc}.
This debate remained one of the largest unsettled debates in physics and astronomy until \citet{Clowe:2004eq} published their studies of the Bullet Cluster (1E 0657-558).
The Bullet Cluster was the first discovered \textit{dissociative merger}, see Figure \ref{fig:MergerTimeSeries} for an example of a dissociative merger.
During the merging process the effectively collisionless galaxies  become dissociated from the collisional intracluster gas, which strongly interacts during the merger and becomes pancaked at the point of collision.
In total the cluster gas is about seven times as massive as the gas.
When \citet{Clowe:2004eq} measured where the total cluster mass was with the gravitational lens effect they found that the majority of the mass was located with the galaxies rather than with the gas.
They then inferred that this was only possible if there was a DM particle that was nearly collisionless like the galaxies; modified gravity could not easily explain such an observation.

\subsection{General Properties of Dark Matter}

For a nice review of our current understanding of DM I recommend \citet{Peter:2012tg}, however I will summarize a few relevant points here.
In addition to the abundance of DM (see \S \ref{section:CompositionOfTheUniverse}) there are a few things we know about the properties of DM.
Most notably DM is electromagnetically neutral.
DM does not interact with photons, either through absorption or emission.
Baryons cannot make up a large portion of DM.
This is know from observations of the cosmic microwave background, large-scale structure of the universe, and abundance of light elements created during big-bang nucleosynthesis.
Nor can DM consist solely of light (sub-keV-mass) particles (e.g. neutrinos). 
Light particles move too fast in the early universe to form the initial density concentrations necessary to seed the cosmic structures observed at later times.
DM is completely outside the realm our Standard Model of Particle Physics.
Thus there is no reason to expect that it only interacts via the known forces.
It is entirely possible that DM interacts with itself via some new dark gauge bosons.
Such DM has been termed self-interacting dark matter (SIDM).

%--------------------------------------------------------------------
\section{Motivation for Studying Self-interacting Dark Matter and Existing Constraints}

\subsection{Early Motivation}

The earliest motivation for studying SIMD came from the \textit{missing satellites problem} \citep[see][for a thorough review]{Bullock:2010uv}.
\citet{Moore:1999ja} and \citet{Klypin:1999ej} were the first to note that the number of Milky Way satellites predicted by $\Lambda$CDM simulations significantly exceeded the number of observed satellites.
In an attempt to resolve this problem \citet{Spergel:2000cb} revived a SIDM model with a large scattering cross-section but with negligible annihilation or dissipation\footnote{Such a SIDM model was first proposed by \citet{Carlson:1992cp} and \citet{Machacek:1994kj} to suppress small scale power of CDM dominated cosmologies.  \citet{deLaix:1995ey} found that while SIDM suppressed the small scale power spectrum in a desirable way it resulted in inconsistencies with the observed properties of galaxies.  The \citet{deLaix:1995ey} objects are no longer valid as they only applied to CDM dominated cosmologies, not $\Lambda$CDM cosmologies.}. 
They found that if the SIDM scattering cross-section ($\sigma_{\rm SIDM}/m_{\rm DM}$) is between 0.45--450\,cm$^2$\,g$^{-1}$ then the expected number of Milky Way satellites could be brought inline with the number of observed satellites.
More recently the Sloan Digital Sky Survey (SDSS) and Sloan Extension for Galactic Understanding and Exploration (SEGUE) have enabled the discovery of a number of new dwarf satellites \citep[see][for a review]{Willman:2010fg}, effectively doubling the number of know satellites.
This in combination with simulations that better quantified the selection function of these surveys has helped to reduce the tension between $\Lambda$CDM and the number of observed satellites.
While the missing satellites problem is not resolved it is not nearly as significant as originally believed.
On a final note, recent simulations \citep{Rocha:2012tr} show that for $\sigma_{\rm SIDM}/m_{\rm DM} \sim 1$\,cm$^2$\,g$^{-1}$ the effects of DM halo evaporation are less than originally estimated by \citet{Spergel:2000cb}'s analytic estimates, especially in the outer radii of parent DM halos where the majority of satellites reside.
%This is in stark contrast to the expectations of warm DM (WDM; $\sim$keV-mass), which was also originally considered as a possible solution of the missing satellites problem. 

\subsection{Early SIDM Constraints}

Following the revival of SIMD by \citet{Spergel:2000cb} a number of researchers began to constrain $\sigma_{\rm SIDM}m_{\rm DM}^{-1}$ through several independent observations, in what \citet{Peter:2012vi} has termed the ``Y2K-era constraints''.
The Y2K-era constraints fall into five categories:
1) Those that compare the central density of simulated SIDM halos with those observed across a range of halo mass-scales from dwarf spheroids to galaxy clusters \citep{Hogan:2000ih, Kochanek:2000iw, Yoshida:2000bd, Yoshida:2000gn, Dave:2001hh, Dalcanton:2001jj, Meneghetti:2001en, Colin:2002ku}.
Some finding that SIDM created central density cores (i.e. flattened density profile) that were too large and others finding that SIDM with large cross-sections actually exacerbates the formation of cusps (i.e. sharply peaked density profile).
2) Those that compare the shape (i.e. spherical or elliptical) of simulated SIDM halos with the shapes of observed DM halos \citep{Yoshida:2000bd, Dave:2001hh, MiraldaEscude:2002ev}.
They found that as the SIDM cross-section is increase DM halos become more spherically symmetric and at a certain point elliptical halos, such as those observed in some galaxy clusters, can no longer be formed.
3)Those that compared the amount of substructure in simulated SIDM halos with the amount observed in galaxies and galaxy clusters \citep{Hogan:2000ih, Yoshida:2000bd, Gnedin:2001gd, Colin:2002ku}.
As the SIDM cross-section increases subhalos begin to evaporate in their parent DM halo.
4) Those that estimated the formation of super massive black holes (SMBH) as a function of varying SIDM cross-section \citep{Hennawi:2002kv}.
As the SIDM cross-section is increased the initial seeds of SMBH's can form earlier and more efficiently, at some point the expected number and mass of SMBH's exceeds the observed number and mass.
5) Those that compared and contrasted the observed behavior of DM with collisionless galaxies and collisional gas during the merging process of two galaxy clusters \citep{Markevitch:2004dl}.
This method will be the focus of this dissertation and is discussed in great detail in \S\ref{section:DMconstraintWithMergers}.

Of these early works four constrained the velocity independent $\sigma_{\rm SIDM}m_{\rm DM}^{-1}$ to such a degree that it became astrophysically uninteresting.
\citet{Gnedin:2001gd} obtained a constraint of $\sigma_{\rm SIDM}m_{\rm DM}^{-1}\lesssim 0.3$\,cm$^2$\,g$^{-1}$ with their study of subhalo evaporation in galaxy clusters.
\citet{Yoshida:2000gn} and \citet{Meneghetti:2001en} obtained a constraint of $\sigma_{\rm SIDM}m_{\rm DM}^{-1}\lesssim 0.1$\,cm$^2$\,g$^{-1}$ with their study of the central densities of galaxy cluster.
Finally \citet{MiraldaEscude:2002ev} obtained tightest constraint, $\sigma_{\rm SIDM}m_{\rm DM}^{-1}\lesssim 0.02$\,cm$^2$\,g$^{-1}$, with their study of galaxy cluster halo shapes.
However recent SIDM simulations \citep{Peter:2012vi, Rocha:2012tr} have cast serious doubts on each of these previous constraints.
However in recent SIDM simulations\citet{Rocha:2012tr} find that the previous subhalo evaporation and central density constraints are likely overestimated, and \citet{Peter:2012vi} points out several weaknesses of the \citet{MiraldaEscude:2002ev} work as well as presents contradictory results.
In summary \citet{Peter:2012vi} and \citet{Rocha:2012tr} find that the previous constraints need to be loosened to $\sigma_{\rm SIDM}m_{\rm DM}^{-1}\lesssim 1.0$\,cm$^2$\,g$^{-1}$.
As will be discussed in \S\ref{section:RecentMotivation}  $\sigma_{\rm SIDM}m_{\rm DM}^{-1}$ between $\sim$0.3--1.0\,cm$^2$\,g$^{-1}$ has potentially interesting and desirable astrophysical implications.

%Central density profile (i.e. cores in halo):
%
%Dwarfs density cores: Dave et al. 2001 sigma/m=0.1--10
%Hogan \& Dalconton 2000: substructure and central density cusps
%Cluster central density: Meneghetti et al. 2001
%Dalconton \& Hogan 2001: cores in dwarfs to clusters of galaxies
%Kochanek \& White 2000:  claim that SIDM exacerbates the formation of cusps in galaxy halos
%Colin et al. 2002: structure and substructure of Milky Way-sized halos, velocity dependent cross-section, found catastrophic formation of cusps
%Yoshida et al. 2000a {Yoshida:2000bd}
%Yoshida et al. 2000b {Yoshida:2000gn} sigma/m < 0.1
%
%halos shapes (i.e. elliptical halos):
%
%Halo Shapes: Miralda-Escudé 2002 sigma/m < 0.02
%Dwarfs density cores: Dave et al. 2001 sigma/m=0.1--10
%Yoshida et al. 2000a {Yoshida:2000bd}
%
%Substructure in DM halos / halo evaporation:
%
%Subhalo evaporation: Gnedin \& Ostriker 2001 sigma/m < 0.3
%Hogan \& Dalconton 2000: substructure and central density cusps
%Colin et al. 2002: structure and substructure of Milky Way-sized halos, velocity dependent cross-section, found catastrophic formation of cusps
%Yoshida et al 2000a {Yoshida:2000bd}
%
%growth of SMBH:
%
%Hennawi \& Ostriker 2002 sigma/m $\sim$0.02 (need to investigate this further
%
%Merging Galaxy Cluster
%Merging Cluster: Randall et al 2008 sigma/m < 0.7--1.25
%
%By comparing DM halo shapes and the substructure there in \citet{Moore:2000ee} found that simulations with $\sigma_{\rm SIDM}m_{\rm DM}^{-1}\sim 10$\,cm$^2$\,g$^{-1}$ resulted in halos with singular isothermal density profiles which ``is inconsistent with galactic rotation curves''\footnote{\citet{Moore:2000ee} don't actually define the $\sigma_{\rm SIDM}m_{\rm DM}^{-1}$ they use in their model. Since they treat SIDM as a non-radiative gas I have estimated the cross-section by comparing the scattering depth of such a gas with that of SIDM with a given cross-section \citep[see for example][]{Markevitch:2004dl}}.
%Thus \citet{Moore:2000ee} effectively placed a constraint of $\sigma_{\rm SIDM}m_{\rm DM}^{-1}\lesssim 10$\,cm$^2$\,g$^{-1}$.
%
%Repeat similarly for other constraints... (perhaps too much detail though) and I could just summarize the methods that provide the tightest constraints.
%
%See the summary by \citet{Peter:2012vi} on page 106.

\subsection{Recent Motivation}\label{section:RecentMotivation}

After the burst of Y2K-era work, the field of SIDM essentially died, with only a few new constraints trickling in \citep{Randall:2008hs, Dawson:2012dl, Merten:2011gu}.
However, the field has recently entered a renaissance with extensive theoretical work {ArkaniHamed:2009gk, Feng:2010kh, Tulin:2012jt, Tulin:2013eo, Ackerman:2009ia, Pospelov:2008di} and simulation work \citep{Peter:2012vi, Rocha:2012tr, Vogelsberger:2012dy, Vogelsberger:2013bb, Zavala:2013iq}.
This renaissance can largely be attributed to three apparent conflicts between astrophysical observations and $\Lambda$CDM:

% Copied from our NSF proposal
{\it (i)} SIDM would explain the dynamics of dwarf spheroidal
galaxies (dSphs) around the Milky Way.
%Figure~\ref{fig-boylankolchin} from \citet{BoylanKolchin12}
%illustrates the problem.  
In $\Lambda$CDM simulations of Milky Way-mass galaxies, each realization has
of order ten subhalos which are denser than any known Milky Way
dSph \citep{BoylanKolchin:2012id}.  Moreover, stellar kinematic data from
the best studied dSph indicate that these galaxies have \emph{cored}
density profiles, in contrast to $\Lambda$CDM predictions that
these dSph should have cuspy DM density
profiles \citep{Navarro:2004hi, Walker:2011eg, Wolf:2012vc}.
%The density is manifest as a high circular
%velocity $V_{\rm circ}$ even at small radii; the total mass is related
%to $V_{\rm circ}$ at large radii so these are also the more massive
%subhalos, excluding the Magellanic Clouds and their analogs.
%These dSphs are predicted by $\Lambda$CDM, but apparently don't exist; if
%they had formed stars we would have detected them, and scenarios in
%which they {\it didn't} form stars but the lower-mass dSphs {\it did}
%are implausible.  
SIDM would reduce the central density of and create cores in the
subhalos, so that the known luminous dSph would live in the most
massive of the Milky Way's subhalos \citep{Rocha:2012tr}.  Otherwise, one
must argue that many of the densest and most massive subhalos failed
to form stars while lower-mass subhalos did form stars and host the
observed dSph.
%the more massive ones are accounted for by observations 
%Note that $\Lambda$CDM also predicts too many
%subhalos, but at lower masses and densities incompleteness and
%failure to form stars are plausible arguments.  
It is the tension between the data and the higher-density simulated
subhalos which most strongly argues for SIDM.  A recent analysis of
SIDM simulations by \citet{Rocha:2012tr} suggests that a SIDM cross
section $\sigma_{\rm SIDM} \approx 0.1-0.5$\,cm$^2$\,g$^{-1}$ would lead to
the correct central density and core size for the Milky Way dSph.

{\it (ii)} Analyses of stellar kinematics in low surface
brightness (LSB) and dwarf galaxies continue to suggest that these
galaxies have central densities and density profiles more consistent
with the presence of dark-matter cores than the $\Lambda$CDM
prediction of cusps \citep{Simon:2005fu, deNaray:2008iz, Oh:2011jd}.  In the same
analysis that found that $\sigma_{\rm SIDM} \approx 0.1-0.5$\,cm$^2$\,g$^{-1}$ would produce DM cores of the right size and
density for the Milky Way dSph, it was found that this range of cross
section also yields appropriate density profiles for LSB and dwarf
galaxies \citep{Rocha:2012tr}.

{\it (iii)} Recent observations of the ellipticity of the surface
densities \citep{Richard:2010bo} and the central density
profiles \citep{Newman:2012wt, Newman:2012tk} of massive galaxy clusters
suggest that a cross section of $\sigma_{\rm SIDM} \approx 1$\,cm$^2$\,g$^{-1}$ is too large \citep{Peter:2012vi, Rocha:2012tr}.
However, cross sections in the range $\sigma_{\rm SIDM} \approx 0.1-0.5$\,cm$^2$\,g$^{-1}$ may be consistent with these observations.  In
particular, after carefully modeling out the baryonic content of a set
of seven galaxy clusters \citet{Newman:2012tk} find that the central
(within 50 kpc) DM density profile is more consistent with a core or
shallow cusp, characteristic of SIDM, than with a $\Lambda$CDM steep
cusp.

These observations suggest that SIDM is astrophysically interesting
(in the sense that it produces measurable changes to dark-matter
halos) if its cross section is velocity independent and in the range $\sigma_{\rm SIDM}m_{\rm DM}^{-1}\sim$0.1-0.5\,cm$^2$\,g$^{-1}$.  A detection of
$\sigma_{\rm SIDM}$ in this range would explain a variety of astrophysical
observations and open a new window to hidden-sector extensions to the
standard model of physics.  Conversely, a nondetection of SIDM by
experiments sensitive to this range would render the SIDM hypothesis
astrophysically uninteresting.
Given the modified existing constraints of $\sigma_{\rm SIDM}m_{\rm DM}^{-1}\lesssim 1.0$\,cm$^2$\,g$^{-1}$ this is a very narrow window of parameter space to explore.

%--------------------------------------------------------------------
\section{Probes of SIDM}

The only way to investigate whether DM self-interacts is through astrophysical observations.

Baryon degeneracy and why merging cluster may be the best probe

%--------------------------------------------------------------------
\section{Merging Galaxy Clusters as Probes of Self-Interacting Dark Matter}

Section text

\subsection{Merging Galaxy Clusters}\label{Section:MergingClusters}

Think about moving some of the Figure \ref{fig:4ConstraintMethods} caption to the body text to make for a more concise caption.

\begin{figure}
\centering
\includegraphics[width=6in]{Chapter1/MergerTimeSeries.png}
\caption{A basic time series leading to a dissociative galaxy cluster merger (assuming the Standard Model of Cosmology).
Two galaxy clusters (N \& S), each consisting of overlapping halos of DM and gas as well as sparsely populated galaxies, begin with an initial physical separation ($t_{\rm 1}$).
Due to the mass of each galaxy cluster they experience a gravitational attraction and accelerate towards one another until eventually they collide ($t_{\rm 1}$--$t_{\rm 4}$).
By convention time $t_{\rm 4}$ is defined as the ``collision''.
Because there is so much space between the galaxies, a strong interaction between any two galaxies is extremely unlikely and they can be treated as effectively collisionless particles.
And since the galaxies of each subcluster have built up momentum they will pass through and begin to separate (this time on the opposite side), $t_{\rm 5}$.
The gas however is more evenly distributed, thus an interaction between gas particles of each subcluster is more likely.
This interactions will convert some of the infall kinetic energy into thermal energy (i.e. the gas of each subcluster will experience ram pressure), the net effect being that the gas halo of each subcluster is slowed with respect to the galaxies and much of it becomes dissociated remains centered between the galaxies of the two subclusters ($t_{\rm 6}$ C).
Much like the galaxies the DM behaves in a nearly collisionless manner and appears largely coincident with the galaxies.
At $t_{\rm 6}$ the galaxy cluster merger is classed as \emph{dissociative merger}.
\label{fig:MergerTimeSeries}}
\end{figure}  

\subsection{Constraining Self-Interacting Dark Matter with Merging Galaxy Clusters}\label{section:DMconstraintWithMergers}

Think about moving some of the Figure \ref{fig:4ConstraintMethods} caption to the body text to make for a more concise caption.

\begin{figure}
\centering
\includegraphics[width=4in]{Chapter1/4ConstraintMethods.png}
\caption{The four means of constraining $\sigma_{\rm SIDM}$ with merging clusters, originally outlined by \citet{Markevitch:2004dl} and \citet{Randall:2008hs}. 
\emph{Upper Left:} If the DM is significantly offset from the gas then the scattering depth of the dark matter ($\tau_{\rm DM}$) must be less than the scattering depth of the gas ($\tau_{\rm gas}$) and an upper limit can be placed on $\sigma_{\rm SIDM}$.
\emph{Upper Right:} If DM self-interacts during the merger, then the velocity of each subcluster will be slowed to some degree.
Thus if the observed velocity ($v_{\rm obs}$) is found to be consistent with the free-fall velocity ($v_{\rm free-fall}$) an upper limit can be placed on $\sigma_{\rm SIDM}$.
If however $v_{\rm obs}$ is significantly less than $v_{\rm free-fall}$, an upper limit could be potentially placed on $\sigma_{\rm SIDM}$.
\emph{Lower Left:} If DM self-interacts during the merger, then some fraction of the DM particles will scatter and become unbound from each subcluster.
Thus the mass-to-light ratio of each subcluster can be compared with the mass-to-light ratio of similar non-merging clusters, and depending on where the mass-to-light ratio of the merger is the same or less than the non-merging clusters' mass-to-light ratio, then respectively an upper limit or lower limit can be placed on $\sigma_{\rm SIDM}$.
\emph{Lower Right:} If the DM self-interacts during the merger, then the DM component of each subcluster will experience an additional drag force and will travel at a slower velocity than the respective subcluster galaxies.
Thus depending on whether a significant offset between the galaxies and DM is or is not observed, then respectively a lower limit or upper limit can be placed on $\sigma_{\rm SIDM}$.
\label{fig:4ConstraintMethods}}
\end{figure}

\subsection{Advantages of Merging Clusters Over other Probes of Self-interacting Dark Matter}

\section{Work Presented in this Dissertation}
Briefly summarize each chapter.