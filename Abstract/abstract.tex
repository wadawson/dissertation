% Context: Why now

\noindent\textbf{Context:} 
The majority ($\sim85$\%) of the matter in the universe is composed of dark matter, a mysterious particle that does not interact via the electromagnetic force yet does interact with all other matter via the gravitational force.
Much work in the past and currently has recently been devoted to finding interactions of dark matter with baryonic matter via the weak force.
To date only tentative and highly controversial evidence for such interactions has been discovered.
While such direct detection experiments have ruled out the possibility that dark matter interacts with baryonic matter via a strong scale force, it is still possible that dark matter interacts with itself via a strong scale force and has a self-scattering cross-section of $\sim0.5$\,cm$^2$/,g$^{-1}$.
In fact such a strong scale scattering force could resolve several outstanding astronomical mysteries:...
 

% Need: What you want is not what you have

\noindent\textbf{Need:} 
While such probes suggest that dark matter may self-scatter, each suffers from a \emph{baryonic degeneracy}, where the observations might be explained by various baryonic processes (e.g. AGN or supernove feedback, XXX etc.), in fact the important scales of these observations often coincide with baryonic scales (e.g. the core size in clusters is approximately the size of the brightest cluster galaxy).
What is needed is a probe of self-interacting dark matter (SIDM) where the expected effect cannot be replicated by the same processes reponsible for the baryonic degeneracy in the aforementioned probes.
Merging galaxy cluster are such a probe.
During the merging process the effectively collisionless galaxies ($\sim$2\% of the cluster mass) become dissociated from the collisional intracluster gas ($\sim$15\% of the cluster mass), which strongly interacts during the merger and becomes pancaked at the point of collision.
If dark matter lags behind the effectively collisionless galaxies then this is clear evidence that dark matter self-interacts.
The expected galaxy-dark matter offset is of order 100\,kpc (for cross-sections that would explain the other aforementioned mysteries), this is considerably larger than the scales of baryonic degeneracy processes.

%Task: What I did to address the need
\noindent\textbf{Task:} 
To test whether dark matter self-interacts  have carried out a comprehensive survey of the dissociative merging galaxy cluster DLSCL J0916.2+2951 (also known as the Musket Ball Cluster):
photometric and spectroscopic observations to quantify the position and velocity of the cluster galaxies,
weak gravitational lensing observations to map and weigh the mass (i.e. dark matter) of the cluster,
Sunyaev-Zel'dovich effect and X-ray observations map and quantify the intracluster gas,
and finally radio observations to search for associated radio relics that could help constrain the properties of the merger.
Using this information in conjunction with a Monte Carlo analysis model I quantify the dynamic properties of the merger, necessary for properly interpreting any constraints on the SIDM cross-section.
I compare the locations of the galaxies, dark matter and gas
This dissertation presents this work.

% Findings: What I found doing the task
\noindent\textbf{Findings:} 
We find that the Musket Ball is a modest merger with total mass of XXX.
However, my dynamic analysis shows that the Musket Ball is being observed XXX\,Gyr after first pass through and is much further progressed in its merger process than previously identified dissociative mergers (XXX times further progressed that the Bullet Cluster). 
By observing that the dark matter is significantly offset from the gas we are able to place an upper limit on the dark matter cross-section of XXX.
We find an offset of XXX between the galaxies and dark matter in the southern subcluster.

% Conclusion: What these findings mean to me
\noindent\textbf{Conclusion:}
This offset suggests that dark matter self-interacts with XXX\% confidence, and appear to be consistent with existing SIDM constraints.
While this offset is significant, it is not significant enough to claim that dark matter self-interacts.

% Perspectives: What we should do next
\noindent\textbf{Perspectives:}
The galaxy-dark matter offset measurement is a Poisson noise dominated measurement.
Thus measuring this offset in other dissociative mergers holds the promise of reducing our uncertainty and enabling us to: 1) state confidently that dark matter self-interacts via a new dark sector force, or 2) constrain the dark matter cross-section to such a degree that SIDM cannot explain the aforementioned mysteries.
To this end we have established the merging cluster collaboration to observe and simulate an ensemble of dissociative merging cluster.
We are currently in the process of analyzing six dissociative mergers with existing data, and carrying out multi-wavelength observations of a new sample of XXX radio relic identified dissociative mergers.