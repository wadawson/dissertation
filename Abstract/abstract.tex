% Context: Why now

\noindent\textbf{Context:} 
The majority ($\sim85$\%) of the matter in the universe is composed of dark matter, a mysterious particle that does not interact via the electromagnetic force yet does interact with all other matter via the gravitational force.
Many direct detection experiments have been devoted to finding interactions of dark matter with baryonic matter via the weak force.
To date only tentative and controversial evidence for such interactions has been found.
While such direct detection experiments have ruled out the possibility that dark matter interacts with baryonic matter via a strong scale force, it is still possible that dark matter interacts with itself via a strong scale force and has a self-scattering cross-section of $\sim0.5$\,cm$^2$\,g$^{-1}$.
In fact such a strong scale scattering force could resolve several outstanding astronomical mysteries: a discrepancy between the cuspy density profiles seen in $\Lambda$CDM simulations and the cored density profiles observed in low surface brightness galaxies, dwarf spheroidal galaxies, and galaxy clusters, as well as the discrepancy between the significant number of massive Milky Way dwarf spheroidal halos predicted by $\Lambda$CDM and the dearth of observed Milky Way dwarf spheroidal halos.
 

% Need: What you want is not what you have

\noindent\textbf{Need:} 
While such observations are in conflict with $\Lambda$CDM and suggest that dark matter may self-scatter, each suffers from a \emph{baryonic degeneracy}, where the observations might be explained by various baryonic processes (e.g., AGN or supernove feedback, stellar winds, etc.)\footnote{At the heart of this is a current lack of knowledge of the influence of baryons on structure formation.} rather than self-interacting dark matter (SIDM).
In fact, the important scales of these observations often coincide with baryonic scales (e.g., the core size in clusters is approximately the half-light radius of the brightest cluster galaxy).
What is needed is a probe of SIDM where the expected effect cannot be replicated by the same processes responsible for the baryonic degeneracy in the aforementioned probes.
Merging galaxy clusters are such a probe.
During the merging process the effectively collisionless galaxies ($\sim$2\% of the cluster mass) become dissociated from the collisional intracluster gas ($\sim$15\% of the cluster mass).
A significant fraction of the gas self-interacts during the merger and slows down at the point of collision.
If dark matter lags behind the effectively collisionless galaxies then this is clear evidence that dark matter self-interacts.
The expected galaxy-dark matter offset is typically $>$25\,kpc (for cross-sections that would explain the other aforementioned issues with $\Lambda$CDM), larger than the scales of that are plagued by the baryonic degeneracies.

%Task: What I did to address the need
\noindent\textbf{Task:} 
To test whether dark matter self-interacts we have carried out a comprehensive survey of the dissociative merging galaxy cluster DLSCL J0916.2+2951 (also known as the Musket Ball Cluster).
This survey includes photometric and spectroscopic observations to quantify the position and velocity of the cluster galaxies,
weak gravitational lensing observations to map and weigh the mass (i.e., dark matter which comprises $\sim$85\% of the mass) of the cluster,
Sunyaev-Zel'dovich effect and X-ray observations to map and quantify the intracluster gas,
and finally radio observations to search for associated radio relics, which had they been observed would have helped constrain the properties of the merger.
Using this information in conjunction with a Monte Carlo analysis model I quantify the dynamic properties of the merger, necessary to properly interpret constraints on the SIDM cross-section.
I compare the locations of the galaxies, dark matter and gas to constrain the SIDM cross-section.
This dissertation presents this work.

% Findings: What I found doing the task
\noindent\textbf{Findings:} 
We find that the Musket Ball is a merger with total mass of  $4.8^{+3.2}_{-1.5}\times 10^{14}$M$_\sun$.
The dynamic analysis shows that the Musket Ball is being observed $1.1^{+1.3}_{-0.4}$\,Gyr after first pass through and is much further progressed in its merger process than previously identified dissociative mergers (for example it is $3.4^{+3.8}_{-1.4}$ times further progressed that the Bullet Cluster). 
By observing that the dark matter is significantly offset from the gas we are able to place an upper limit on the dark matter cross-section of $\sigma_{\rm SIDM}m^{-1}_{\rm DM} <$8\,cm$^2$\,g$^{-1}$.
Furthermore, we find an that the galaxies appear to be leading the weak lensing (WL) mass distribution by 20.5'' (129\,kpc at $z$=0.53) in southern subcluster, which might be expected to occur if dark matter self-interacts.
Contrary to this finding though the WL mass centroid appears to be leading the galaxy centroid by 7.4'' (47\,kpc at $z$=0.53) in the northern subcluster. 


% Conclusion: What these findings mean to me
\noindent\textbf{Conclusion:}
The southern offset alone suggests that  dark matter self-interacts with $\sim$83\% confidence.
However, when we account for the observation that the galaxy centroid appears to trail the WL centroid in the north the confidence falls to $\sim$55\%.
While the SIDM scenario is slightly preferred over the CDM scenario it is not significantly so.

% Perspectives: What we should do next
\noindent\textbf{Perspectives:}
The galaxy-dark matter offset measurement is dominated by random errors in each cluster.
Thus measuring this offset in other dissociative mergers holds the promise of reducing our uncertainty and enabling us to: 1) state confidently whether dark matter self-interacts via a new dark sector force, or 2) constrain the dark matter cross-section to such a degree that SIDM cannot explain the aforementioned mysteries\footnote{In the case of a null detection of an offset between the galaxies and the DM, SIDM simulations will be necessary to place a quantitative constraint on the SIDM cross-section \citep[see e.g.,][]{Randall:2008hs}.}.
To this end we have established the Merging Cluster Collaboration to observe and simulate an ensemble of dissociative merging clusters.
We are currently in the process of analyzing six dissociative mergers with existing data, and carrying out multi-wavelength observations of a new sample of 15 radio relic identified dissociative mergers.